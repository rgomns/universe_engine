\documentclass[11pt]{article}

\usepackage{amsmath, amssymb, amsthm}
\usepackage{bbm}
\usepackage{fullpage}
\usepackage{hyperref}
\usepackage{enumitem}

\title{\textbf{A General Theory of Law Ecology}}
\author{Roy [Surname]}
\date{2026}

\theoremstyle{definition}
\newtheorem{definition}{Definition}

\theoremstyle{plain}
\newtheorem{theorem}{Theorem}
\newtheorem{lemma}{Lemma}
\newtheorem{corollary}{Corollary}

\theoremstyle{remark}
\newtheorem{remark}{Remark}

\begin{document}

\maketitle

\begin{abstract}
We develop a formal framework in which laws are modeled as adaptive constraints evolving within a dynamical universe. A universe is represented as a population of microstates in a finite configuration space, and laws are represented as constraint operators with strengths that evolve through a discrete-time stochastic update rule. Under mild mixing assumptions, the induced constraint graph undergoes a percolation transition that forces the system into a low-entropy attractor sector. We prove the existence of Lyapunov-like descent in expectation, establish conditions for collapse into a finite set of pointer sectors, and show that the resulting dynamics form a universality class independent of microscopic details. Quantum measurement appears as a restricted special case of this broader ecological mechanism. Law ecology is presented as a meta-theory of law formation rather than a proposal for new microscopic physics.
\end{abstract}

\section{Introduction}

Traditional physical theories treat laws as fixed axioms. In contrast, complex adaptive systems exhibit emergent regularities that behave like evolving constraints. This motivates a general framework in which laws are endogenous objects subject to variation, reinforcement, and decay.

We formalize this idea by defining a \emph{law ecology}: a dynamical system in which constraints evolve alongside the microstates they regulate. This framework unifies concepts from statistical physics, learning theory, constraint satisfaction, and quantum foundations.

\section{Axioms of Law Ecology}

Let $\mathcal{X} = \{0,1\}^n$ denote the configuration space.

\begin{definition}[Universe]
A universe at time $t$ is a pair $(\mathcal{U}_t, C_t)$ where:
\begin{itemize}[noitemsep]
    \item $\mathcal{U}_t \subseteq \mathcal{X}$ is a population of size $N$,
    \item $C_t$ is a finite set of constraints.
\end{itemize}
\end{definition}

\begin{definition}[Empirical Distribution]
The empirical distribution is


\[
\rho_t(x) = \frac{1}{N} \left|\{u \in \mathcal{U}_t : u = x\}\right|.
\]


\end{definition}

\begin{definition}[Constraint]
A constraint is a triple $c = (S, f, s)$ where:
\begin{itemize}[noitemsep]
    \item $S \subseteq \{1,\dots,n\}$ is a scope,
    \item $f : \{0,1\}^{|S|} \to \{0,1\}$ indicates forbidden patterns,
    \item $s \ge 0$ is a strength.
\end{itemize}
\end{definition}

\begin{definition}[Energy Functional]
Given $C_t$, define


\[
E_{C_t}(x) = \sum_{c \in C_t} s_c f_c(x|_{S_c}).
\]


\end{definition}

\begin{definition}[Stochastic Update Rule]
There exists a Markov kernel


\[
\Phi : (\rho_t, C_t) \mapsto (\rho_{t+1}, C_{t+1})
\]


driven by randomness $\eta_t$, implementing:
\begin{enumerate}[noitemsep]
    \item emergence of new constraints from rarity statistics,
    \item relaxation toward lower energy,
    \item reinforcement of satisfied constraints,
    \item decay of unsatisfied constraints,
    \item bounded noise.
\end{enumerate}
\end{definition}

\section{Emergence of Laws}

\begin{definition}[Rarity-Based Emergence]
For a randomly sampled scope $S$, let $\rho_{t,S}$ be the empirical distribution over patterns.  
Define the forbidden pattern:


\[
f(p) = \mathbbm{1}\{p = \arg\min_q \rho_{t,S}(q)\}.
\]


\end{definition}

This defines a mutation operator on $C_t$.

\section{Constraint Graph and Percolation}

\begin{definition}[Skeleton Graph]
The skeleton graph is


\[
G_{C_t} = (V, E_t), \qquad (i,j) \in E_t \iff \exists c \in C_t : i,j \in S_c.
\]


\end{definition}

\begin{theorem}[Dominance of Erd\H{o}s--R\'enyi Under Mixing]
Assume:
\begin{enumerate}[noitemsep]
    \item scopes $S$ are sampled uniformly,
    \item $\rho_t$ is sufficiently mixing,
    \item constraint emergence is rarity-based.
\end{enumerate}
Then $G_{C_t}$ stochastically dominates an Erd\H{o}s--R\'enyi graph $G(n,p_t)$ for some $p_t > 0$.
\end{theorem}

\begin{corollary}[Percolation Transition]
If $p_t n > 1$, then $G_{C_t}$ contains a giant connected component.
\end{corollary}

\section{Collapse Dynamics}

\begin{definition}[Pointer Sectors]


\[
\mathcal{P} = \arg\min_{x \in \mathcal{X}} E_{C_t}(x).
\]


\end{definition}

\begin{theorem}[Lyapunov Descent in Expectation]
There exists a function $L(\rho_t, C_t)$ such that


\[
\mathbb{E}[L(\rho_{t+1}, C_{t+1}) \mid \rho_t, C_t] \le L(\rho_t, C_t).
\]


\end{theorem}

\begin{theorem}[Collapse into Finite Pointer Sectors]
Under percolation of $G_{C_t}$ and bounded noise,


\[
|\mathcal{P}| = O(1).
\]


\end{theorem}

\section{Universality}

\begin{theorem}[Universality Class of Collapse]
Any law ecology satisfying:
\begin{itemize}[noitemsep]
    \item finite configuration space,
    \item rarity-based emergence,
    \item reinforcement,
    \item bounded noise,
    \item percolation of $G_{C_t}$,
\end{itemize}
belongs to the same universality class of collapse dynamics.
\end{theorem}

\section{Scaling Behavior}

\begin{theorem}[Collapse Time Scaling]
Let $t_c(n)$ be the expected time to percolation.  
Under random scope sampling,


\[
t_c(n) = O(\log n).
\]


\end{theorem}

\section{Embedding Quantum Mechanics}

Quantum mechanics can be embedded as a restricted law ecology in which:
\begin{itemize}[noitemsep]
    \item constraints are linear operators,
    \item strengths correspond to fixed amplitudes,
    \item reinforcement is unitary evolution,
    \item collapse is projection onto eigenspaces.
\end{itemize}

Thus QM is a special case with fixed, non-evolving laws.

\section{Relation to Existing Frameworks}

Law ecology connects to:
\begin{itemize}[noitemsep]
    \item spin glasses (energy landscapes),
    \item SAT percolation (constraint density transitions),
    \item Hopfield networks (Hebbian reinforcement),
    \item decoherence theory (pointer sectors).
\end{itemize}

\section{Ontology of Laws}

Laws in this framework are population-level statistical regularities encoded as constraints.  
They are not metaphysical primitives but emergent structures influencing dynamics.

\section{Conclusion}

Law ecology provides a meta-theory of law formation, generalizing constraint satisfaction and collapse mechanisms. It unifies classicality emergence, attractor dynamics, and adaptive constraint evolution under a single mathematical structure.

\bibliographystyle{plain}
\begin{thebibliography}{9}

\bibitem{zurek}
W.~H. Zurek.
\newblock Decoherence, einselection, and the quantum origins of the classical.
\newblock \emph{Rev. Mod. Phys.}, 75:715--775, 2003.

\bibitem{mezard}
M.~M\'ezard, G.~Parisi, and M.~Virasoro.
\newblock \emph{Spin Glass Theory and Beyond}.
\newblock World Scientific, 1987.

\bibitem{hopfield}
J.~Hopfield.
\newblock Neural networks and physical systems with emergent collective computational abilities.
\newblock \emph{PNAS}, 79(8):2554--2558, 1982.

\end{thebibliography}

\end{document}
